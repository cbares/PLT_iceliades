\documentclass[a4paper,12pt]{report}

\usepackage{textcomp}
\usepackage[utf8]{inputenc}
\usepackage[francais]{babel}
\usepackage{graphicx}
\usepackage[francais]{layout}
\usepackage[hidelinks]{hyperref}
\usepackage{subcaption}

\DeclareGraphicsExtensions{.png}

%%%%%%



%%%%%%

\begin{document}


	\begin{figure}
		\includegraphics[width = 0.2 \textwidth]{image_latex/logo.png}
	\end{figure}

	
	\title{\textbf{Projet Zorglub}}
	\author{\hline \\ Charlet Justine, Descatoire Theau \\ Montmaur Antoine,  Sathiakumar Kuga \\ \\ \hline \\  \includegraphics[width = 0.7 \textwidth]{image_latex/logo_zorglub.png} }
	
	
	\date{\textbf{ 6 Octobre 2020}}

	
	
	
	
	
	
	\maketitle
	\tableofcontents
	
	
	
	
	
	
	
	
	
	
	\chapter{Environnement de développement}
	\section{Introduction}
	Le Projet Logiciel Transverse (PLT) est un projet de 120h destiné à la conception d'un jeu vidéo de type tour par tour, avec un réseau multi joueur en ligne. Il fait intervenir la plupart des cours dispensés en Informatique et Systèmes à l'ENSEA : génie logiciel, algorithmique, programmation parallèle et web services. \\
	\indent Au cours de ce projet, nous allons faire de la conception UML, réaliser un moteur d'affichage, créer une IA basique, une heuristique puis l'améliorer pour qu'elle soit performante, et enfin développer ce jeu en multijoueur, de façon à pouvoir jouer en ligne.


	
	\section{Outils pour la réalisation du jeu}
	
		Ce projet est développé en langage C++, très adapté pour les jeux vidéos. Nous avons décidé de travailler soit sous VM, soit sur un PC avec un environnement Linux. Le projet est développé sous Ubuntu 16.04. Nous avons décidé de coder sous CLION et d'utiliser les bibliothèques SFML pour le rendu. \\ \\

		Nous avons créé un projet GIT en clonant le projet du GIT de M. Barès et nous avons créé nos 4 branches personnelles.
		De plus, nous avons crée un projet Trello afin de mieux visualiser les tâches restantes.
	
	
	
	
	
	\section{Archétype du jeu}
	L'objectif de notre projet est la réalisation d'un jeu vidéo de type tactical RPG (T-RPG), en tour par tour, basé sur le jeu Donjon de Naheulbeuk. 
	
	\begin{figure}[h]
		\centering
		\includegraphics[width = 0.5\columnwidth]{image_latex/donjon.png}
		\caption{Logo de Donjon de Naheulbeuk}
	\end{figure}
	
	
	
	\section{Règles du jeu}
	
	
	\subsection{Les personnages et leurs maps}
	Nous avons créé nos propres personnages et nos propres règles.\\
	Chaque joueur doit choisir une équipe de 3 personnages parmi les suivants : Elfe, Indien, Nain, Bandit, Chevalier, Troll et Pirate.  \\

	\indent Le jeu comporte 7 maps représentant les 7 thèmes associés aux personnages : 
	\begin{enumerate}
		\item Jungle (elfe)
		\item Plaine (indien)
		\item Montagne (nain)
		\item Forêt (bandit)
		\item Château (chevalier)
		\item Grotte (troll)
		\item Bateau (pirate)
	\end{enumerate}	
	. \\ 
	
	
	
	
	
	\begin{figure*}[h]
		\centering
		\begin{subfigure}[b]{0.2\linewidth}
			\centering
			\includegraphics[width=\textwidth]{image_latex/elfe.png}
			\caption{Elfe} 
		\end{subfigure}
		\begin{subfigure}[b]{0.2\linewidth}
			\centering
			\includegraphics[width=\textwidth]{image_latex/indien.png}
			\caption{Indien}
		\end{subfigure}
		\begin{subfigure}[b]{0.2\linewidth}
			\centering
			\includegraphics[width=\textwidth]{image_latex/nain.png}
			\caption{Nain}
		\end{subfigure}
		
		
		\begin{subfigure}[b]{0.2\linewidth}
			\centering
			\includegraphics[width=\textwidth]{image_latex/bandit.png}
			\caption{Bandit} 
		\end{subfigure}
		\begin{subfigure}[b]{0.2\linewidth}
			\centering
			\includegraphics[width=\textwidth]{image_latex/chevalier.png}
			\caption{Chevalier}
		\end{subfigure}
		\begin{subfigure}[b]{0.2\linewidth}
			\centering
			\includegraphics[width=\textwidth]{image_latex/troll.png}
			\caption{Troll}
		\end{subfigure}
		
		
		\begin{subfigure}[b]{0.2\linewidth}
			\centering
			\includegraphics[width=\textwidth]{image_latex/pirate.png}
			\caption{Pirate}
		\end{subfigure}

		\caption{Liste des personnages possibles}
	\end{figure*}



	. \\ \\ 
	
	\subsection{Les attributs principaux, secondaires et les compétences}
	

	\indent A chaque personnage est associé des attributs principaux et secondaires. La répartition des points pour chaque attribut principal se fait par un système d'achat de points:  

	\begin{center}
		\begin{tabular}{|c|}
			
			\hline
			Attributs principaux \\
			\hline
			$8 \le Force \le 15$ \\
			\hline
			$8 \le Adresse \le 15$ \\
			\hline
			$8 \le Endurance \le 15$ \\
			\hline
			$8 \le Courage \le 15$ \\
			\hline
			$8 \le Intelligence \le 15$ \\
			\hline
			$8 \le Arcane \le 15$ \\
			\hline
			
		\end{tabular}
	\end{center}
	. \\ 
	Tous les attributs principaux sont à 8 par défaut et le joueur dispose de 27 points pour améliorer les capacités de son choix, chaque capacité n'excédant pas 15. Si un attribut dépasse 13, il faut alors 2 pts pour passer à 14, puis à 15. \\
	
	Les points attribués aux attributs secondaires sont calculés automatiquement en fonction des attributs primaires de la manière suivante : \\
	
	\begin{tabular}{|c|c|}
		\hline
		\multicolumn{2}{|c|}{\textbf{Attributs secondaires}} \\
		\hline
		Points De Vie & 3 * Endurance + 2 * Force \\
		\hline
		Precision & (Adresse + Force + Intelligence + Arcane)/60 \\
		\hline
		Esquive & ((Adresse + Intelligence - 16 )$^2$/196) * 0,33 \\
		\hline
		Mouvement & 5 + $\lfloor$ 1/12 $\rfloor$ * endurance + $\lfloor$ 1/12 $\rfloor$ * Courage \\
		\hline
		Initiative & Courage + Intelligence + Arcane \\
		\hline
	\end{tabular}
	
	.\\ 
	
	
	Il est aussi possible d'utiliser les compétences des personnages, établies comme suit : \\ 

	\begin{tabular}{|c|c|c|} 
		
		\hline
		\textbf{Corps-à-corps} & \textbf{A distance} & \textbf{Magie} \\
		\hline
		- 2 pt d'actions (ts ls 3 tours) & Rebonds (ts ls 2 tours) &Téléportation (1) \\
		\textit{Le joueur 2 ne joue pas} & \textit{L'attaque rebondit 2 fois} &  \\
		\hline
		Pas de mouvement (2)& Pluie de projectiles (2) & Désarmement (2) \\
		& \textit{Joueur 2 essuie 3 attaques} & \\
		\hline
	\end{tabular}
	. \\
	
	Chaque compétence vaut 1 point d'action et ne dure qu'un tour. Ces compétences ont un temps de rechargement exprimé en nombre de tours (voir tableau ci-dessus). \\ \\
	\textbf{Bonus thématique}\\
	De plus, si un personnage joue dans la map de même thème (ex : Elfe dans la jungle), alors le personnage se voit attribuer un bonus de 5 points de vie. \\ \\
	\textbf{Bonus Taxon} \\ 
	De même, chaque personnage a un bonus lié à son origine, selon le tableau ci-dessous: \\ \\
	
	
	\begin{tabular}{|c|c|}
		\hline
		\multicolumn{2}{|c|}{Bonus de Taxon} \\
		\hline
		Elfe &  +1 en Adresse \\
		\hline
		Indien & +1 en Arcane \\
		\hline
		Nain & +1 en Endurance \\
		\hline
		Bandit & +1 en  Intelligence\\
		\hline
		Troll & +1 en Force \\
		\hline
		Chevalier & +1 en Courage \\
		\hline
		Pirate & +1 au choix dans un attribut primaire (car c'est un voleur)\\
		\hline
	\end{tabular}

	
	
	
	
	
	\subsection{Le tour par tour} 
	A chaque tour, le joueur dispose de 2 points d'action. Pour chaque point, il peut soit se déplacer, soit attaquer, soit utiliser une compétence. Pour chaque tour, il n'est pas possible de réaliser une attaque et une compétence. \\ \\ \\
	\textbf{\underline{Les attaques}}\\ \\
	Trois types d'attaques sont disponibles: le corps-à-corps, attaque à distance, ou attaque magique.\\ \\ \\ Pour chaque arme, une portée minimale et maximale est donnée dans le tableau ci dessous : \\ \\ \\
	
	
	
	\begin{tabular}{|c|c|c|c|c|c|c|c|c|}
		\hline
		\multicolumn{9}{|c|}{Types d'attaques} \\
		\hline 
		\multicolumn{3}{|c|}{ \textbf{Corps-à-corps}} & 
		\multicolumn{3}{|c|}{ \textbf{A distance}}  & \multicolumn{3}{|c|}{\textbf{Magique}}\\
		\hline
		& \multicolumn{2}{|c|}{\textbf{Portée}} & & \multicolumn{2}{|c|}{\textbf{Portée}} & & \multicolumn{2}{|c|}{\textbf{Portée}} \\
		\hline
		\textbf{Arme} & Min & Max & \textbf{Arme}& Min & Max & \textbf{Arme}& Min & Max\\
		\hline 
		Epée & 1 & 2 & Arc & 2 & 8 & Baguette & 3 & 4,5 \\
		\hline
		Hache & 1 & 1,5 & Arbalette & 2 & 7 & Bâton & 3 & 4\\
		\hline
		Lance & 1 & 2,5 & Fronde & 2 & 6,5 &  Bracelets & 3 & 5 \\
		\hline

	\end{tabular}
	. \\ \\ \\
	Voici les dégâts causés par ces armes : \\ 
	\begin{center}
		\begin{tabular}{|c|c|}
			
			\hline
			Arme & Dégâts \\
			\hline
			Epée & 13 \\
			Hache & 14\\
			Lance & 12\\
			Arc & 6\\
			Arbalette & 7 \\
			Fronde & 8 \\
			Baguette & 10 \\
			Bâton & 11 \\
			Bracelet & 9 \\
			\hline
		\end{tabular}
	\end{center}
	.\\ \\
	\textbf{Les déplacements}\\
	
	Chaque personnage peut se déplacer de 5 unités de distance: selon la verticale et/ou horizontale pour une valeur de 1 unité par case, et en diagonale pour une valeur de 1,5 unité de distance par case. Le joueur peut également avoir un bonus de 1 ou 2 unité de distance s'il choisit de maxer l'attribut secondaire "Mouvement".\\ Exemple:
	\begin{figure*}[h]
		\centering
		\begin{subfigure}[b]{0.4\linewidth}
			\centering
			\includegraphics[width=\textwidth]{image_latex/hv.png}
			\caption{4 cases de distance} 
		\end{subfigure}
		\begin{subfigure}[b]{0.4\linewidth}
			\centering
			\includegraphics[width=\textwidth]{image_latex/diag.png}
			\caption{3,5 cases (1,5+2)}
		\end{subfigure}
	\end{figure*}
	
	\newpage
	
	\section{Diagramme}
	\subsection{Diagramme d'états}
	
	\begin{figure}[h!]
		\centering
		\includegraphics[width = 1.34\columnwidth, angle = 90]{image_latex/diaetat.png}
		
	\end{figure}
	\newpage
	\subsection{Diagramme de classes}
	\begin{figure}[h!]
		\centering
		\includegraphics[width = 1.42\columnwidth, angle = 90]{image_latex/diaclasse.png}
		
	\end{figure}

	\begin{figure}[h!]
		\centering
		\includegraphics[width = 1.8\columnwidth, angle = 90]{image_latex/diaclasse2.png}
		
	\end{figure}
	
\end{document}
